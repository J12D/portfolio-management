\documentclass{scrreprt}

\usepackage{jdcommands}
\jdfontsetup{}

\usepackage{amsmath, amssymb}
\usepackage[USenglish]{babel}

\usepackage{pgfplots}
\pgfplotsset{compat=newest}
\usepgfplotslibrary{dateplot}
\usepackage{pgfcalendar}

\usepackage{wrapfig}
\usepackage{subcaption}
\usepackage{booktabs}
\usepackage{float}
\usepackage{listings}
\lstset{numbers=left, numberstyle=\tiny, frame=lines, breaklines, caption=\lstname}


\usetikzlibrary{external}
\tikzexternalize[prefix=tikz/]
%\tikzset{external/system call={lualatex -shell-escape -halt-on-error -interact
%ion=batchmode -jobname "\image" "\texsource"}}

\definecolor{dbblue}{RGB}{0,24,168}
\definecolor{lightblue}{RGB}{0,152,219}
\definecolor{warmgrey}{RGB}{152,143,134}
\definecolor{orange}{RGB}{255,161,0}
\definecolor{green}{RGB}{122,184,0}
\definecolor{palegrey}{RGB}{224,230,230}

\title{Portfolios for a safe future}
\author{JD POWER}
\begin{document}
\maketitle

\pgfplotsset{y label style={at={(axis description cs:-0.08,.5)}, text depth=0.25ex, anchor=south,}}

\pgfplotsset{every axis legend/.append style={
        at={(0.5,1.03)},
        anchor=south}}

\pgfplotsset{
  /pgfplots/bar cycle list/.style={/pgfplots/cycle list={%
      {fill=lightblue, lightblue!80!black},
      {fill=warmgrey, warmgrey!80!black},
      {fill=orange, orange!80!black},
      {fill=green, green!80!black},
      {fill=palegrey, palegrey!80!black},
  }
},
}

\pgfplotsset{
  gen/.style={
      no marks,
      date coordinates in=x,
      xticklabel=\year,
      xtick={2008-01-01,2009-01-01,2010-01-01,2011-01-01,2012-01-01, 2013-01-01, 2014-01-01, 2015-01-01},
      yticklabel style={/pgf/number format/fixed,/pgf/number format/precision=2},
      legend pos = north west,
      every axis plot/.append style={thick, smooth},
      width=\textwidth,
      height=0.5\textwidth,
      legend pos = north west,
      legend columns=5,
      every axis legend/.append style={at={(0.5,1.03)}, anchor=south},
      trim axis left,trim axis right,
      stack plots=y,
      cycle list name=exotic,
      area style,
      every axis/.append style={font=\small},
      legend style={font=\small},
  },
  total/.style={red, ultra thick, line legend},
   relw/.style 2 args={
      gen,
      ylabel=,
      yticklabel=\pgfmathparse{100*\tick}\pgfmathprintnumber{\pgfmathresult}\,\%,
      ymin=#1, ymax=#2,
    },
   w/.style 2 args={
      gen,
      ylabel=,
      yticklabel=\pgfmathparse{\tick}\pgfmathprintnumber{\pgfmathresult}\,\%,
      ymin=#1, ymax=#2,
    },
  bottom/.style={axis y line = none},
  cycle list name=exotic,
  assetplot/.style={
    no marks,
    date coordinates in = x,
    xticklabel = \year,
    xtick = {2009-01-01,2011-01-01,2013-01-01,2015-01-01},
    yticklabel style = {/pgf/number format/fixed, /pgf/number format/precision=3},
    yticklabel = \pgfmathparse{\tick}\pgfmathprintnumber{\pgfmathresult}\,\%,
    width = 0.5\textwidth,
    height = 0.3\textwidth,
    trim axis left,
    trim axis right,
    every axis plot/.append style={thick, smooth, dbblue},
  }
}

\chapter{Become active regarding your wealth}
First we would like to introduce ourselves and JD investments.
JD investments is one of the world’s leading investment organizations with an asset under management of over EUR 132 billion.
The company offers institutions traditional and alternative investments across all major asset classes.
We offer our clients first-rate product management and a consistently superior client service platform.
 
We are JD Investments’ team of most experienced portfolio managers.
With over 30 years of market experience, global expertise and one of the smallest but most successful asset management team for active asset allocation and risk mitigation in Europe.
The following factors make us specialists in our industry: 

\begin{itemize}
\item[$\checkmark$] Diversified active portfolio manager
\item[$\checkmark$] Global investment and research capabilities 
\item[$\checkmark$] Consultative local delivery 
\item[$\checkmark$] Forward looking investment strategies 
\end{itemize}

We would like to guide you through our investment process and advise you with an investment that is tailored to your individual needs.
You came to us as an experienced investor looked to expand your investment portfolio with the following five assets: DAX, Dow Jones Industrial, Nikkei 225, iPath S\&P 500 and the risk free rate (Euribor).
We have compiled an overview of various portfolio strategies that would fit to your risk mentality.
Furthermore we back-tested these strategies, in order to see which of these would fit best to your return expectations.
  
In our first interaction you have set some limitations in our investment strategy which we took into account in order to create the portfolios customized to your needs.
You stated that you do not wish high transaction costs, are relatively risk averse and want maximum return.
Therefore we have set up constraints to limit turnover to 10+-2\%, p.
a., limit drawdown to less than 10\% p.a. and maximize realized Sharpe ratio.

In the following proposal we will start off by introducing the asset classes you have selected and our view on them.
Afterwards we will give a brief summary of the current market environment that is monitored by our highly-renown economic analysts.
The proposal includes a detailed description of our investment philosophy, since it is the key to understanding how we have optimized and which tools we have used for our portfolios.
This is followed by a summary of the portfolios we have custom-made for you and ultimately our personal recommendation.
 


\chapter{Asset Universe}
%Because investor is risk-averse, we invest in indexes of mature markets.

\tikzsetnextfilename{assets_dax}
\begin{wrapfigure}{r}{0.5\textwidth}
  \centering
  \begin{tikzpicture}
    \begin{axis}[assetplot]
      \addplot[dbblue] table[x=time, y=value.DAX] {plot-data/assets_vxx.txt};
    \end{axis}
  \end{tikzpicture}
\end{wrapfigure}
%
\paragraph{DAX} is a German capitalization-weighted stock market index that includes the top 30 (blue chip) German companies trading on the Frankfurt Stock Exchange.
The DAX is considered to be one of the most important indices for European investors, since the German economy is the fourth largest in the world/the largest in Europe.
Furthermore, many DAX companies are heavily export oriented, giving an investor broad exposure to global markets. In fact, more than 75\% of DAX revenues stem from international markets, indicating a sound international diversification.

\tikzsetnextfilename{assets_dj}
\begin{wrapfigure}{r}{0.5\textwidth}
  \centering
  \begin{tikzpicture}
    \begin{axis}[assetplot]
      \addplot[dbblue] table[x=time, y=value.Dow.Jones] {plot-data/assets_vxx.txt};
    \end{axis}
  \end{tikzpicture}
\end{wrapfigure}
%
\paragraph{Dow Jones Industrial Average} is a price-weighted (in USD) stock market index that describes the stock performance of 30 large publicly owned companies based in the United States.
The index proved itself as very stable, able to overcome large market downturns, since most companies in the index are well-established and internationally diversified.
Given America's status as the world's leading economy, the Dow Jones offers investors the chanece to participate in the growth of the most influenatial companies in the US.

\tikzsetnextfilename{assets_nikkei}
\begin{wrapfigure}{r}{0.5\textwidth}
  \centering
  \begin{tikzpicture}
    \begin{axis}[assetplot]
      \addplot[dbblue] table[x=time, y=value.Nikkei] {plot-data/assets_vxx.txt};
    \end{axis}
  \end{tikzpicture}
\end{wrapfigure}
%
\paragraph{Nikkei 225} is a price-weighted (in Yen) stock market index for the Tokyo Stock Exchange and is considered as the best average representative of the Japanese Market.
The Japanese economy is the world’s third largest and is assumed to enter a recovery phase after after two decades of stagnation.
Japan's economy leads all world economies in various measures that are testament for its impressive potential for innovation, such as R\&D as proportion of GDP, number of registered patents and environment for the development of business.
Moreover, Japan is considered as mature market gateway to quickly growing Asian market.


\tikzsetnextfilename{assets_vxx}
\begin{wrapfigure}{r}{0.5\textwidth}
  \centering
  \begin{tikzpicture}
    \begin{axis}[assetplot]
      \addplot[dbblue] table[x=time, y=value.VXX] {plot-data/assets_vxx.txt};
    \end{axis}
  \end{tikzpicture}
\end{wrapfigure}
%
\paragraph{iPath S\&P 500 VIX Short-Term Futures ETN} is made of medium-term notes, created by Barclays Bank PLC that are uncollateralized debt securities, performance of which is highly related to the performance of the underlying VIX futures (S&P 500 VIX ShortTerm Futures™ Index TR) with maturities 1 and 2 months.
The main benefit of VXX is that by buying it investor gets exposure to VIX futures but does not have to constantly roll over these futures that produces additional costs.  VXX is highly correlated to VIX index, with correlation almost equal to 1.
VIX benchmark index reflects market opinion and corresponding price of volatility of S\&P 500 Index over 30 days in the future.


Normally, in times of uncertainty the value of the index increases, so investor can benefit from going long VIX futures when realized volatility appears to be larger than implied. Short positions in the VIX are taken when investor expects market implied volatility to decrease in the future. Moreover, historically VIX was negatively correlated with the underlying index. That is, in 80\% of time VIX and S\&P 500 moved in opposite direction. Therefore, when markets fall, long position in VIX futures should increase in value. All this is also true for VXX, but compared to futures strategies (with constant rolling over), here we just take long or short position of VXX in our portfolio and slightly adjust it with time.

 
\paragraph{Euribor} (Euro Interbank Offered Rate) is an estimate rate at which European banks borrow funds from one another.
It is calculated by excluding as a truncated mean of quoted rates.
As of 10th November, the Euribor rate was -0.004\%, averaging 0\% throughout November.

\section*{Our Benchmark: MSCI}
In order to show that we have a good performance we need to compare our performance with a benchmark.. Since we invest in indices of developed countries, the MSCI can be used as a benchmark. The MSCI world is an index  which includes medium and large capitalization stocks of 23 developed markets. Concerning the performance of the index in our out-of-sample period starting from 1st January 2013 till 30th June 2015 the annual return of the index was 20.597\%, annual volatility -17.867\% and corresponding Sharpe ratio –1.15.


\chapter*{Portfolios}

\section*{1 -- JD Investments Minimum Variance Equity Fund}
\textit{JD Investments Minimum Variance Equity} fund invests in American, German and Japanese equity market together with 6M Euribor to substantially reduce risk.
The asset allocation is oriented on volatility minimization while disregarding expected returns.
This portfolio is particularly suitable for the clients with strong risk-aversion, since it does not try to speculate on expected returns but rather cares just about reducing variance.

\paragraph{Portfolio Construction}
\begin{sit}
\item Long only equity exposure
\item Risk reduction through investments in Euribor
\item High concentration in risk-free investment (45\% of portfolio)
\item No assumption on expected returns
\end{sit}

\tikzsetnextfilename{pf_1}
\begin{figure}[H]
\begin{tikzpicture}
  \begin{axis}[w={0}{140}, y filter/.expression={max(y,0)}]
    \addplot table[x=time, y=value.DAX] {plot-data/minv.txt}\closedcycle;
    \addlegendentry{DAX}
    \addplot table[x=time, y=value.Dow.Jones] {plot-data/minv.txt}\closedcycle;
    \addlegendentry{Dow Jones}
    \addplot table[x=time, y=value.Nikkei] {plot-data/minv.txt}\closedcycle;
    \addlegendentry{Nikkei}
    \pgfplotsset{cycle list shift=1}
    \addplot table[x=time, y=value.mixed] {plot-data/minv.txt}\closedcycle;
    \addlegendentry{Euribor}
    \addlegendimage{total} \addlegendentry{Total Value}
  \end{axis}
  \begin{axis}[w={0}{140}, stack plots = false]
    \addplot[total] table[x=time, y=value] {plot-data/minv_p.txt};
    %\addplot[black] table[x=time, y=value] {plot-data/msci.txt};
  \end{axis}
  \begin{axis}[w={0}{140}, y filter/.expression={min(y,0)}]
    \addplot table[x=time, y=value.DAX] {plot-data/minv.txt}\closedcycle;
    \addplot table[x=time, y=value.Dow.Jones] {plot-data/minv.txt}\closedcycle;
    \addplot table[x=time, y=value.Nikkei] {plot-data/minv.txt}\closedcycle;
    \pgfplotsset{cycle list shift=4}
    \addplot table[x=time, y=value.mixed] {plot-data/minv.txt}\closedcycle;
  \end{axis}
\end{tikzpicture}
\end{figure}

\tikzsetnextfilename{pf_1w}
\vspace{\abovedisplayskip}
\begin{minipage}{0.65\textwidth}
\begin{tikzpicture}
  \begin{axis}[relw={0}{1}, y filter/.expression={max(y,0)}]
    \addplot table[x=time, y=value.DAX] {plot-data/minv_relw.txt}\closedcycle;
    \addlegendentry{DAX}
    \addplot table[x=time, y=value.Dow.Jones] {plot-data/minv_relw.txt}\closedcycle;
    \addlegendentry{Dow Jones}
    \addplot table[x=time, y=value.Nikkei] {plot-data/minv_relw.txt}\closedcycle;
    \addlegendentry{Nikkei}
    \pgfplotsset{cycle list shift=1}
    \addplot table[x=time, y=value.mixed] {plot-data/minv_relw.txt}\closedcycle;
    \addlegendentry{Euribor}
  \end{axis}
  \begin{axis}[relw={0}{1},  y filter/.expression={min(y,0)}]
    \addplot table[x=time, y=value.DAX] {plot-data/minv_relw.txt}\closedcycle;
    \addplot table[x=time, y=value.Dow.Jones] {plot-data/minv_relw.txt}\closedcycle;
    \addplot table[x=time, y=value.Nikkei] {plot-data/minv_relw.txt}\closedcycle;
    \pgfplotsset{cycle list shift=1}
    \addplot table[x=time, y=value.mixed] {plot-data/minv_relw.txt}\closedcycle;
  \end{axis}
\end{tikzpicture}
\end{minipage}
\begin{minipage}{0.35\textwidth}
\begin{tabular}{lr}
\toprule
Ann. Return & 10.92\%\\
Ann. Volatility & 7.99\%\\
Sharpe Ratio & 1.37\\
Max. Drawdown & 5.92\% \\
Turnover & 11.56\%\\
\bottomrule
\end{tabular}
\end{minipage}

This portfolio is suitable for very risk-averse clients who are ready to sacrifice potential returns and have lower Sharpe ratio just to reduce volatility.

\begin{table}[H]
\begin{tabularx}{\textwidth}{XX}
  \toprule
  \textbf{\textsf{Opportunities}} & \textbf{\textsf{Risks}} \\
  \midrule
Possibility to gain attractive returns through participation American, German and Japanese stock markets.&
Risks of obtaining very low returns in the periods of high concentration in Euribor since  return on short-term Euribor is often 0 and sometimes even negative. \\[1em]
Risks of indexes are substantially reduced with the help of mixture with risk-free asset. &
Losses possible in case unhedged currency position develop against exposure built up in the portfolio.\\
  \bottomrule
\end{tabularx}
\end{table}


\newpage\section*{2 -- Max Sharpe}
The Maximum Sharpe Ratio portfolio is constructed with the use of sophisticated mathematical techniques. The objective of the fund is to deliver significant risk adjusted return, and to outperform any other portfolio within a similar investment universe. 
At any point in time, assets of the portfolio are invested fully into risky assets. It is suitable for enthusiastic risk seekers than are willing to take high risk to achieve significant returns. As shown in the figure below, majority of funds are invested into US market and the rest is allocated in Germany and Japan. From the beginning of 2013 to the middle of 2015, MSR portfolio delivered return of over 17\% annually, while volatility was moderate and amounted to 12.05\%. 

\tikzsetnextfilename{pf_2}
\begin{figure}[H]
\begin{tikzpicture}
  \begin{axis}[w={0}{170}, y filter/.expression={max(y,0)}]
    \addplot table[x=time, y=value.DAX] {plot-data/ms.txt}\closedcycle;
    \addlegendentry{DAX}
    \addplot table[x=time, y=value.Dow.Jones] {plot-data/ms.txt}\closedcycle;
    \addlegendentry{Dow Jones}
    \addplot table[x=time, y=value.Nikkei] {plot-data/ms.txt}\closedcycle;
    \addlegendentry{Nikkei}
    \addlegendimage{total} \addlegendentry{Total Value}
  \end{axis}
  \begin{axis}[w={0}{170}]
    \addplot[total] table[x=time, y=value] {plot-data/ms_p.txt};
  \end{axis}
  \begin{axis}[w={0}{170}, y filter/.expression={min(y,0)}]
    \addplot table[x=time, y=value.DAX] {plot-data/ms.txt}\closedcycle;
    \addplot table[x=time, y=value.Dow.Jones] {plot-data/ms.txt}\closedcycle;
    \addplot table[x=time, y=value.Nikkei] {plot-data/ms.txt}\closedcycle;
  \end{axis}
\end{tikzpicture}
\end{figure}

\tikzsetnextfilename{pf_2w}
\vspace{\abovedisplayskip}
\begin{minipage}{0.65\textwidth}
  \begin{tikzpicture}
  \begin{axis}[relw={0}{1}, y filter/.expression={max(y,0)}]
    \addplot table[x=time, y=value.DAX] {plot-data/ms_relw.txt}\closedcycle;
    \addlegendentry{DAX}
    \addplot table[x=time, y=value.Dow.Jones] {plot-data/ms_relw.txt}\closedcycle;
    \addlegendentry{Dow Jones}
    \addplot table[x=time, y=value.Nikkei] {plot-data/ms_relw.txt}\closedcycle;
    \addlegendentry{Nikkei}
  \end{axis}
  \begin{axis}[relw={0}{1}, y filter/.expression={min(y,0)}]
    \addplot table[x=time, y=value.DAX] {plot-data/ms_relw.txt}\closedcycle;
    \addplot table[x=time, y=value.Dow.Jones] {plot-data/ms_relw.txt}\closedcycle;
    \addplot table[x=time, y=value.Nikkei] {plot-data/ms_relw.txt}\closedcycle;
  \end{axis}
\end{tikzpicture}
\end{minipage}
\begin{minipage}{0.35\textwidth}
\begin{tabular}{lr}
\toprule
Ann. Return & 17.55\%\\
Ann. Volatility & 12.05\%\\
Sharpe Ratio & 1.46 \\
Max. Drawdown & 8.92\% \\
Turnover & 11.94\%\\
\bottomrule
\end{tabular}
\end{minipage}



\begin{table}[H]
\begin{tabularx}{\textwidth}{XX}
  \toprule
  \textbf{\textsf{Opportunities}} & \textbf{\textsf{Risks}} \\
  \midrule
Active participation in the growth of US, German and Japanese stock market. Investor profits from momentum in the stock market. &
Portfolio exposed to sudden increase in volatility. \\[1em]
Portfolio rebalanced and optimized on yearly basis. By optimizing, investor achieves “best” asset allocation given investment set. &
Due to 100\% allocation in risky assets, investor exposed to liquidity shock in the market.\\
  \bottomrule
\end{tabularx}
\end{table}

\newpage\section*{3 -- JD Investments Fixed Weight Equity Fund}
\textit{JD Investments Fixed Weight Equity Fund} invests in American, German and Japanese equity markets.
The asset allocation is static and most diversified while disregarding expected returns and variances/covariance’s.
This portfolio is especially appealing since it is resistant against errors in true variance/covariance and expected return estimation.

\paragraph{Portfolio Construction}
\begin{sit}
\item	Long only and full equity exposure
\item	Fixed asset allocation
\item	Strong diversification
\item	No assumption on expected returns and variances/covariances
\end{sit}

\tikzsetnextfilename{pf_3}
\begin{figure}[H]
\begin{tikzpicture}
  \begin{axis}[w={0}{170}]
    \addplot table[x=time, y=value.DAX] {plot-data/fw.txt}\closedcycle;
    \addlegendentry{DAX}
    \addplot table[x=time, y=value.Dow.Jones] {plot-data/fw.txt}\closedcycle;
    \addlegendentry{Dow Jones}
    \addplot table[x=time, y=value.Nikkei] {plot-data/fw.txt}\closedcycle;
    \addlegendentry{Nikkei}
    \addlegendimage{total} \addlegendentry{Total Value}
  \end{axis}
  \begin{axis}[w={0}{170}]
    \addplot[total] table[x=time, y=value] {plot-data/fw_p.txt};
  \end{axis}
\end{tikzpicture}
\end{figure}

\tikzsetnextfilename{pf_3w}
\vspace{\abovedisplayskip}
\begin{minipage}{0.65\textwidth}
\begin{tikzpicture}
  \begin{axis}[relw={0}{1}]
    \addplot table[x=time, y=value.DAX] {plot-data/fw_relw.txt}\closedcycle;
    \addlegendentry{DAX}
    \addplot table[x=time, y=value.Dow.Jones] {plot-data/fw_relw.txt}\closedcycle;
    \addlegendentry{Dow Jones}
    \addplot table[x=time, y=value.Nikkei] {plot-data/fw_relw.txt}\closedcycle;
    \addlegendentry{Nikkei}
  \end{axis}
\end{tikzpicture}
\end{minipage}
\begin{minipage}{0.35\textwidth}
\begin{tabular}{lr}
\toprule
Ann. Return & 17.17\%\\
Ann. Volatility & 12.00\%\\
Sharpe Ratio & 1.48 \\
Max. Drawdown & 8.92 \\
Turnover & 0\\
\bottomrule
\end{tabular}
\end{minipage}
\vspace{\belowdisplayskip}

With a promising track record and a solid risk return profile as well as zero turnover and therefore very appealing net cost return the \textit{JD Investments Fixed Weight Equity Fund} is an attractive investment for the investor with a moderate risk profile. 

\begin{table}[H]
\begin{tabularx}{\textwidth}{XX}
  \toprule
  \textbf{\textsf{Opportunities}} & \textbf{\textsf{Risks}} \\
  \midrule
Prospect of positive return through participation American, German and Japanese stock markets &
Losses possible in case stock markets develop against exposure built up in the portfolio.\\[1em]
Prospect of positive returns due to unhedged currency exposure. &
Losses possible in case unhedged currency position develop against exposure built up in the portfolio.\\
  \bottomrule
\end{tabularx}
\end{table}

\newpage\section*{4 -- Equal Weight Leveraged}
EWL portfolio offers investor asset allocation into three different markets, namely US, German and Japanese. Weights in the portfolio are equally distributed with each asset class consisting of 36\% of total portfolio value. Furthermore, investor’s return will be enhanced by shorting VXX index that is predicted to have a bear run in the near future. Portfolio has no holdings in riskless assets, and invests heavily into diversified market indices. As shown in the figure below, portfolio has performed extremely well over the past two years. In this period EWL portfolio achieved return of more than 21\% with considerably lower volatility. 

\tikzsetnextfilename{pf_4}
\begin{figure}[H]
\begin{tikzpicture}
  \begin{axis}[w={-20}{190}, y filter/.expression={max(y,0)}]
    \addplot table[x=time, y=value.DAX] {plot-data/fw_lev.txt}\closedcycle;
    \addlegendentry{DAX}
    \addplot table[x=time, y=value.Dow.Jones] {plot-data/fw_lev.txt}\closedcycle;
    \addlegendentry{Dow Jones}
    \addplot table[x=time, y=value.Nikkei] {plot-data/fw_lev.txt}\closedcycle;
    \addlegendentry{Nikkei}
    \addplot table[x=time, y=value.VXX.Adjusted] {plot-data/fw_lev.txt}\closedcycle;
    \addlegendentry{VXX}
    \addlegendimage{total} \addlegendentry{Total Value}
  \end{axis}
  \begin{axis}[w={-20}{190}]
    \addplot[total] table[x=time, y=value] {plot-data/fw_lev_p.txt};
  \end{axis}
  \begin{axis}[w={-20}{190}, bottom, y filter/.expression={min(y,0)}]
    \pgfplotsset{cycle list shift=3}
    \addplot table[x=time, y=value.VXX.Adjusted] {plot-data/fw_lev.txt}\closedcycle;
  \end{axis}
\end{tikzpicture}
\end{figure}

\tikzsetnextfilename{pf_4w}
\vspace{\abovedisplayskip}
\begin{minipage}{0.65\textwidth}
\begin{tikzpicture}
  \begin{axis}[relw={-0.1}{1.2}, y filter/.expression={max(y,0)}]
    \addplot table[x=time, y=value.DAX] {plot-data/fw_lev_relw.txt}\closedcycle;
    \addlegendentry{DAX}
    \addplot table[x=time, y=value.Dow.Jones] {plot-data/fw_lev_relw.txt}\closedcycle;
    \addlegendentry{Dow Jones}
    \addplot table[x=time, y=value.Nikkei] {plot-data/fw_lev_relw.txt}\closedcycle;
    \addlegendentry{Nikkei}
    \addplot table[x=time, y=value.VXX.Adjusted] {plot-data/fw_lev_relw.txt}\closedcycle;
    \addlegendentry{VXX}
  \end{axis}
  \begin{axis}[relw={-0.1}{1.2}, bottom, y filter/.expression={min(y,0)}]
    \pgfplotsset{cycle list shift=3}
    \addplot table[x=time, y=value.VXX.Adjusted] {plot-data/fw_lev_relw.txt}\closedcycle;
  \end{axis}
\end{tikzpicture}
\end{minipage}
\begin{minipage}{0.35\textwidth}
\begin{tabular}{lr}
\toprule
Ann. Return & 21.47\%\\
Ann. Volatility & 13.95\%\\
Sharpe Ratio & 1.54 \\
Max. Drawdown & 9.83\% \\
Turnover & 0\\
\bottomrule
\end{tabular}
\end{minipage}

\begin{table}[H]
\begin{tabularx}{\textwidth}{XX}
  \toprule
  \textbf{\textsf{Opportunities}} & \textbf{\textsf{Risks}} \\
  \midrule
Active participation in the growth of US, German and Japanese stock market. Investor profits low volatility in the stock market. &
Portfolio is extremely exposed to sudden increase in volatility, since it is short volatility index.\\[1em]
Portfolio is not rebalanced and thus there are no extra fees for buying or selling securities. &
Maximal drawdown has to be carefully monitored due to the running of short position.\\
  \bottomrule
\end{tabularx}
\end{table}

\newpage\section*{5 -- JD Investments Black Litterman Equity Fund}
JD Investments Black Litterman Equity fund invests in American, German and Japanese equity market.
The asset allocation is static and based on the market information combined with our opinion concerning expected returns.
This portfolio is very attractive in terms of returns since it is built with the help of the predictions of our highly qualified analysts concerning prospective development of the markets.

\paragraph{Portfolio Construction}
\begin{sit}
\item Long only and full equity exposure
\item Market optimal asset allocation adjusted with regards to our views
\item Deep analysis of indexes to get own estimation of expected returns and variances/covariances
\end{sit}

\tikzsetnextfilename{pf_5}
\begin{figure}[H]
\begin{tikzpicture}
  \begin{axis}[w={0}{170}, y filter/.expression={max(y,0)}]
    \addplot table[x=time, y=value.DAX] {plot-data/bl.txt}\closedcycle;
    \addlegendentry{DAX}
    \addplot table[x=time, y=value.Dow.Jones] {plot-data/bl.txt}\closedcycle;
    \addlegendentry{Dow Jones}
    \addplot table[x=time, y=value.Nikkei] {plot-data/bl.txt}\closedcycle;
    \addlegendentry{Nikkei}
    \addlegendimage{total} \addlegendentry{Total Value}
  \end{axis}
  \begin{axis}[w={0}{170}]
    \addplot[total] table[x=time, y=value] {plot-data/bl_p.txt};
  \end{axis}
\end{tikzpicture}
\end{figure}

\tikzsetnextfilename{pf_5w}
\vspace{\abovedisplayskip}
\begin{minipage}{0.65\textwidth}
\begin{tikzpicture}
  \begin{axis}[relw={0}{1}, y filter/.expression={max(y,0)}]
    \addplot table[x=time, y=value.DAX] {plot-data/bl_relw.txt}\closedcycle;
    \addlegendentry{DAX}
    \addplot table[x=time, y=value.Dow.Jones] {plot-data/bl_relw.txt}\closedcycle;
    \addlegendentry{Dow Jones}
    \addplot table[x=time, y=value.Nikkei] {plot-data/bl_relw.txt}\closedcycle;
    \addlegendentry{Nikkei}
  \end{axis}
\end{tikzpicture}
\end{minipage}
\begin{minipage}{0.35\textwidth}
\begin{tabular}{lr}
\toprule
Ann. Return & 17.71\%\\
Ann. Volatility & 12.00\%\\
Sharpe Ratio & 1.48\\
Max. Drawdown & 8.92\% \\
Turnover & 0\%\\
\bottomrule
\end{tabular}
\end{minipage}

\begin{table}[H]
\begin{tabularx}{\textwidth}{XX}
  \toprule
  \textbf{\textsf{Opportunities}} & \textbf{\textsf{Risks}} \\
  \midrule
Prospect of high positive return through participation analysis of all indexes and allocation of funds based partially on this analysis. &
Not protected against all the markets crashing simultaneously. \\[1em]
Prospect of positive returns due to unhedged currency exposure. &
Losses possible in case unhedged currency position develop against exposure built up in the portfolio.\\
  \bottomrule
\end{tabularx}
\end{table}

\newpage\section*{6 -- Equal Risk Contribution}
The idea of ER portfolio is to design weights in such a way that each risky asset class contributes to the risk of a portfolio in the same proportion.
Portfolio is rebalanced every month in order to carefully monitor risk contributions to the portfolio.
Portfolio is appealing to risk averse investors who care more about low risk than high returns.
Investor will mainly profit from higher than anticipated return on less risky assets in the portfolio.
As it can be seen in the figure below, portfolio value is increasing smoothly without any significant drawdowns over the period from 2013 to the middle of 2015, and has yielded return of 13.68\%.


\tikzsetnextfilename{pf_6}
\begin{figure}[H]
\begin{tikzpicture}
  \begin{axis}[w={0}{170}, y filter/.expression={max(y,0)}]
    \addplot table[x=time, y=value.DAX] {plot-data/erc.txt}\closedcycle;
    \addlegendentry{DAX}
    \addplot table[x=time, y=value.Dow.Jones] {plot-data/erc.txt}\closedcycle;
    \addlegendentry{Dow Jones}
    \addplot table[x=time, y=value.Nikkei] {plot-data/erc.txt}\closedcycle;
    \addlegendentry{Nikkei}
    \pgfplotsset{cycle list shift=1}
    \addplot table[x=time, y=value.mixed] {plot-data/erc.txt}\closedcycle;
    \addlegendentry{Euribor}
    \addlegendimage{total} \addlegendentry{Total Value}
  \end{axis}
  \begin{axis}[w={0}{170}]
    \addplot[total] table[x=time, y=value] {plot-data/erc_p.txt};
  \end{axis}
\end{tikzpicture}
\end{figure}

\tikzsetnextfilename{pf_6w}
\vspace{\abovedisplayskip}
\begin{minipage}{0.65\textwidth}
\begin{tikzpicture}
  \begin{axis}[relw={0}{1}]
    \addplot table[x=time, y=value.DAX] {plot-data/erc_relw.txt}\closedcycle;
    \addlegendentry{DAX}
    \addplot table[x=time, y=value.Dow.Jones] {plot-data/erc_relw.txt}\closedcycle;
    \addlegendentry{Dow Jones}
    \addplot table[x=time, y=value.Nikkei] {plot-data/erc_relw.txt}\closedcycle;
    \addlegendentry{Nikkei}
    \pgfplotsset{cycle list shift=1}
    \addplot table[x=time, y=value.mixed] {plot-data/erc_relw.txt}\closedcycle;
    \addlegendentry{Euribor}
  \end{axis}
\end{tikzpicture}
\end{minipage}
\begin{minipage}{0.35\textwidth}
\begin{tabular}{lr}
\toprule
Ann. Return & 13.68\%\\
Ann. Volatility & 9.16\%\\
Sharpe Ratio & 1.49 \\
Max. Drawdown & 7.53\% \\
Turnover & 11.53\\
\bottomrule
\end{tabular}
\end{minipage}

\begin{table}[H]
\begin{tabularx}{\textwidth}{XX}
  \toprule
  \textbf{\textsf{Opportunities}} & \textbf{\textsf{Risks}} \\
  \midrule
Active participation in the growth of US, German and Japanese stock market. Investor profits from momentum in the stock market. &
Portfolio exposed to sudden increase in volatility. \\[1em]
Extremely high diversification of the portfolio combined with riskless asset promise investor stable and risk averse portfolio. &\\
  \bottomrule
\end{tabularx}
\end{table}

\newpage\section*{7 --- JD Investments Minimum Variance (Robust)}

\textit{JD Investments Minimum Variance (Robust) Equity} fund invests in American, German and Japanese equity market. The asset allocation is aimed at reducing volatility while variances/covariance matrix is estimated using robust covariance-shrinkage method. This portfolio still provides attractive returns and sharpe ratio in spite of being oriented on variance minimization.

\tikzsetnextfilename{pf_7}
\begin{figure}[H]
\begin{tikzpicture}
  \begin{axis}[w={0}{170}, y filter/.expression={max(y,0)}]
    \addplot table[x=time, y=value.DAX] {plot-data/minv_rob.txt}\closedcycle;
    \addlegendentry{DAX}
    \addplot table[x=time, y=value.Dow.Jones] {plot-data/minv_rob.txt}\closedcycle;
    \addlegendentry{Dow Jones}
    \addplot table[x=time, y=value.Nikkei] {plot-data/minv_rob.txt}\closedcycle;
    \addlegendentry{Nikkei}
    \addlegendimage{total} \addlegendentry{Total Value}
  \end{axis}
  \begin{axis}[w={0}{170}]
    \addplot[total] table[x=time, y=value] {plot-data/minv_rob_p.txt};
  \end{axis}
\end{tikzpicture}
\end{figure}

\tikzsetnextfilename{pf_7w}
\begin{minipage}{0.65\textwidth}
\begin{tikzpicture}
\begin{axis}[relw={0}{1}]
  \addplot table[x=time, y=value.DAX] {plot-data/minv_rob_relw.txt}\closedcycle;
  \addlegendentry{DAX}
  \addplot table[x=time, y=value.Dow.Jones] {plot-data/minv_rob_relw.txt}\closedcycle;
  \addlegendentry{Dow Jones}
  \addplot table[x=time, y=value.Nikkei] {plot-data/minv_rob_relw.txt}\closedcycle;
  \addlegendentry{Nikkei}
\end{axis}
\end{tikzpicture}
\end{minipage}
\begin{minipage}{0.5\textwidth}
\begin{tabular}{lr}
  \toprule
  Ann. Return & 18.10\%\\
  Ann. Volatility & 12.32\%\\
  Sharpe & 1.47\\
  Max. Drawdown& 8.2\%\\
  Alpha & ---\\
  Ann Return Net & \\
  \bottomrule
\end{tabular}
\end{minipage}

\begin{table}[H]
\begin{tabularx}{\textwidth}{XX}
  \toprule
  \textbf{\textsf{Opportunities}} & \textbf{\textsf{Risks}} \\
  \midrule
Prospect of positive return through participation American, German and Japanese stock markets despite the minimum variance goal. &
High concentration in Dow Jones which reduces diversification benefits seriously.\\[1em]
Prospect of positive returns due to unhedged currency exposure. &
Losses possible in case unhedged currency position develop against exposure built up in the portfolio.\\
  \bottomrule
\end{tabularx}
\end{table}

%\newpage\section*{8 --- Max Sharpe Robust}
%\begin{tikzpicture}
  %\begin{axis}[w={0}{170}, y filter/.expression={max(y,0)}]
    %\addplot table[x=time, y=value.DAX] {plot-data/msr_rob.txt}\closedcycle;
    %\addlegendentry{DAX}
    %\addplot table[x=time, y=value.Dow.Jones] {plot-data/msr_rob.txt}\closedcycle;
    %\addlegendentry{Dow Jones}
    %\addplot table[x=time, y=value.Nikkei] {plot-data/msr_rob.txt}\closedcycle;
    %\addlegendentry{Nikkei}
    %\addlegendimage{total} \addlegendentry{Total Value}
  %\end{axis}
  %\begin{axis}[w={0}{170}]
    %\addplot[total] table[x=time, y=value] {plot-data/msr_rob_p.txt};
  %\end{axis}
%\end{tikzpicture}

%\begin{tikzpicture}
  %\begin{axis}[relw={0}{1}]
    %\addplot table[x=time, y=value.DAX] {plot-data/msr_rob_relw.txt}\closedcycle;
    %\addlegendentry{DAX}
    %\addplot table[x=time, y=value.Dow.Jones] {plot-data/msr_rob_relw.txt}\closedcycle;
    %\addlegendentry{Dow Jones}
    %\addplot table[x=time, y=value.Nikkei] {plot-data/msr_rob_relw.txt}\closedcycle;
    %\addlegendentry{Nikkei}
  %\end{axis}
%\end{tikzpicture}

\newpage\section*{9 --- Mean Variance with Maximum Allocation Constraint}
\begin{figure}[H]
\begin{tikzpicture}
  \begin{axis}[w={-10}{120}, y filter/.expression={max(y,0)}]
    \addplot table[x=time, y=value.DAX] {plot-data/mva.txt}\closedcycle;
    \addlegendentry{DAX}
    \addplot table[x=time, y=value.Dow.Jones] {plot-data/mva.txt}\closedcycle;
    \addlegendentry{Dow Jones}
    \addplot table[x=time, y=value.Nikkei] {plot-data/mva.txt}\closedcycle;
    \addlegendentry{Nikkei}
    \addplot table[x=time, y=value.VXX.Adjusted] {plot-data/mva.txt}\closedcycle;
    \addlegendentry{VXX}
    \addplot table[x=time, y=value.mixed] {plot-data/mva.txt}\closedcycle;
    \addlegendentry{Euribor}
    \addlegendimage{total} \addlegendentry{Total Value}
  \end{axis}
  \begin{axis}[w={-10}{120}]
    \addplot[total] table[x=time, y=value] {plot-data/mva_p.txt};
  \end{axis}
  \begin{axis}[w={-10}{120}, bottom, y filter/.expression={min(y,0)}]
    \pgfplotsset{cycle list shift=3}
    \addplot table[x=time, y=value.VXX.Adjusted] {plot-data/mva.txt}\closedcycle;
  \end{axis}
\end{tikzpicture}
\end{figure}

\begin{tikzpicture}
  \begin{axis}[relw={-0.1}{1.1}, y filter/.expression={max(y,0)}]
    \addplot table[x=time, y=value.DAX] {plot-data/mva_relw.txt}\closedcycle;
    \addlegendentry{DAX}
    \addplot table[x=time, y=value.Dow.Jones] {plot-data/mva_relw.txt}\closedcycle;
    \addlegendentry{Dow Jones}
    \addplot table[x=time, y=value.Nikkei] {plot-data/mva_relw.txt}\closedcycle;
    \addlegendentry{Nikkei}
    \addplot table[x=time, y=value.VXX.Adjusted] {plot-data/mva_relw.txt}\closedcycle;
    \addlegendentry{VXX}
    \addplot table[x=time, y=value.mixed] {plot-data/mva_relw.txt}\closedcycle;
    \addlegendentry{Euribor}
  \end{axis}
  \begin{axis}[relw={-0.1}{1.1}, bottom, y filter/.expression={min(y,0)}]
    \pgfplotsset{cycle list shift=3}
    \addplot table[x=time, y=value.VXX.Adjusted] {plot-data/mva_relw.txt}\closedcycle;
  \end{axis}
\end{tikzpicture}

\newpage\section*{10 --- JD Investments Mean Variance with Maximum Allocation Constraint}
\textit{JD Investments Mean Variance with Maximum Allocation Constraint} Equity fund invests in American, German and Japanese equity market alongside with investment in VXX and Euribor to optimize risk allocation. The asset allocation is dynamic and highly concentrated in Euribor. This portfolio is the safest portfolio out of all alternatives in terms of volatility.
%
\paragraph{Portfolio Construction}
\begin{sit}
\item	Equity Indexes, Euribor 6M, VXX
\item	75\% invested in Euribor
\item	Monthly rebalancing
\item	Maximum weight for each index is 50\% relative to whole risky portfolio
\item	Expected returns and variances/covariances are adjusted using shrinkage methods
\end{sit}


\tikzsetnextfilename{pf_10}
\begin{figure}[H]
\begin{tikzpicture}
  \begin{axis}[w={0}{120}, y filter/.expression={max(y,0)}, legend columns=3]
    \addplot table[x=time, y=value.DAX] {plot-data/mvo.txt}\closedcycle;
    \addlegendentry{DAX}
    \addplot table[x=time, y=value.Dow.Jones] {plot-data/mvo.txt}\closedcycle;
    \addlegendentry{Dow Jones}
    \addplot table[x=time, y=value.Nikkei] {plot-data/mvo.txt}\closedcycle;
    \addlegendentry{Nikkei}
    \addplot table[x=time, y=value.VXX.Adjusted] {plot-data/mvo.txt}\closedcycle;
    \addlegendentry{VXX}
    \addplot table[x=time, y=value.mixed] {plot-data/mvo.txt}\closedcycle;
    \addlegendentry{Euribor}
    \addlegendimage{total} \addlegendentry{Total Value}
  \end{axis}
  \begin{axis}[w={0}{120}]
    \addplot[total] table[x=time, y=value] {plot-data/mvo_p.txt};
  \end{axis}
  \begin{axis}[w={0}{120}, bottom, y filter/.expression={min(y,0)}]
    \pgfplotsset{cycle list shift=3}
    \addplot table[x=time, y=value.VXX.Adjusted] {plot-data/mvo.txt}\closedcycle;
  \end{axis}
\end{tikzpicture}
\end{figure}

\tikzsetnextfilename{pf_10w}
\begin{minipage}{0.65\textwidth}
\begin{tikzpicture}
  \begin{axis}[relw={0}{1}, legend columns=3]
    \addplot table[x=time, y=value.DAX] {plot-data/mvo_relw.txt}\closedcycle;
    \addlegendentry{DAX}
    \addplot table[x=time, y=value.Dow.Jones] {plot-data/mvo_relw.txt}\closedcycle;
    \addlegendentry{Dow Jones}
    \addplot table[x=time, y=value.Nikkei] {plot-data/mvo_relw.txt}\closedcycle;
    \addlegendentry{Nikkei}
    \addplot table[x=time, y=value.VXX.Adjusted] {plot-data/mvo_relw.txt}\closedcycle;
    \addlegendentry{VXX}
    \addplot table[x=time, y=value.mixed] {plot-data/mvo_relw.txt}\closedcycle;
    \addlegendentry{Euribor}
  \end{axis}
\end{tikzpicture}
\end{minipage}
\begin{minipage}{0.35\textwidth}
\begin{tabular}{lr}
  \toprule
  Ann. Return & 4.12\%\\
  Ann. Volatility & 3.15\%\\
  Sharpe & 1.31\\
  Max. Drawdown& 3.05\%\\
  Turnover & 10.41\%\\
  Alpha & ---\\
  Ann. Return Net & \\
  \bottomrule
\end{tabular}
\end{minipage}

This portfolio is aimed at those who do not tolerate almost any risk. Along with high concentration in risk-free assets, portfolio is also strengthened with VXX derivatives, which reduce potential risks even further in case of increased market volatility.

\begin{table}[H]
\begin{tabularx}{\textwidth}{XX}
  \toprule
  \textbf{\textsf{Opportunities}} & \textbf{\textsf{Risks}} \\
  \midrule
  \bottomrule
\end{tabularx}
\end{table}

\chapter*{Out Of Sample: 2013-01-01 --- 2015-06-30}
\begin{table}[ht]
\centering
\begin{tabular}{lrrrrrrr}
\toprule
                      & Sharpe & $\mu$ &$\sigma$& MDD & T/O & TC & $\alpha$ \\ 
\midrule
Minimum Variance & 1.37 & 10.92 & 7.99 & 5.29 & 11.56 & 0.06 & -0.56 \\ 
  Maximum Sharpe & 1.46 & 17.55 & 12.05 & 8.98 & 11.94 & 0.06 & -0.65 \\ 
  Fixed Weights & 1.42 & 17.96 & 12.61 & 9.08 & 0.00 & 0.00 &  \\ 
  Fixed Weights Leveraged & 1.54 & 21.47 & 13.95 & 9.83 & 0.00 & 0.00 &  \\ 
  Black Litterman & 1.48 & 17.71 & 12.00 & 8.92 & 0.00 & 0.00 &  \\ 
  Equal Risk & 1.49 & 13.68 & 9.16 & 7.53 & 11.53 & 0.06 & 0.33 \\ 
  Min Variance Robust & 1.47 & 18.10 & 12.32 & 8.20 & 9.47 & 0.05 & -0.66 \\ 
  Max Sharpe Robust & 1.12 & 16.55 & 14.82 & 7.70 & 44.45 & 0.22 & -0.49 \\ 
  MV Max Allocation & 1.31 & 4.12 & 3.15 & 3.05 & 10.41 & 0.05 & -0.45 \\ 
  MV Max Allocation \& Short-Constraint & 1.40 & 12.34 & 8.79 & 6.09 & 12.10 & 0.06 & -0.67 \\ 
\bottomrule
\end{tabular}
\end{table}

\tikzsetnextfilename{comparison}
\begin{tikzpicture}
  \begin{axis}[w={0}{180}, stack plots = false, ymin = 95, legend columns = 3]
    \addplot[total, red] table[x=time, y=value] {plot-data/fw_p.txt};
    \addlegendentry{Fixed Weights}
    \addplot[total, blue] table[x=time, y=value] {plot-data/bl_p.txt};
    \addlegendentry{Black Litterman}
    \addplot[total, green] table[x=time, y=value] {plot-data/erc_p.txt};
    \addlegendentry{Equal Risk}
    \addplot[total, orange] table[x=time, y=value] {plot-data/minv_rob_p.txt};
    \addlegendentry{Minimum Variance Robust}
    \addplot[total, black] table[x=time, y=value] {plot-data/mvo_p.txt};
    \addlegendentry{Max Allocation No Short Sale}
    \addplot[total, gray] table[x=time, y=value] {plot-data/msci.txt};
    \addlegendentry{MSCI}
  \end{axis}
\end{tikzpicture}



\chapter*{Bonus: 2015-06-30 --- 2015-11-01}
\begin{table}[ht]
\centering
\begin{tabular}{lrrrrrr}
  \toprule
 & Sharpe & $\mu$ & $\sigma$ & Inf. Ratio\\
  \midrule
Minimum Variance & -1.32 & -15.51 & 11.76 & 0.93\\ 
  Maximum Sharpe & -1.14 & -21.09 & 18.51 & -0.42\\ 
  Fixed Weights & -1.20 & -22.38 & 18.64 & -0.82\\ 
  Fixed Weights Leveraged & -1.27 & -25.50 & 20.02 & -0.94\\ 
  Black Litterman & -1.20 & -22.38 & 18.64 & -0.79\\ 
  Equal Risk & -0.81 & -11.30 & 13.90 & 0.62\\ 
  Min Variance Robust & -1.38 & -25.32 & 18.34 & -0.58\\ 
  Max Sharpe Robust & -1.21 & -22.29 & 18.38 & -0.58\\ 
  MV Max Allocation & -1.03 & -3.87 & 3.74 & 1.52\\ 
  MV Max Allocation \& Short-Constraint & -1.03 & -3.93 & 3.80 & 0.69\\ 
   \bottomrule
\end{tabular}
\end{table}

\tikzsetnextfilename{comparison_bonus}
\begin{tikzpicture}
  \begin{axis}[w={80}{110}, stack plots = false, xtick={2015-07-01, 2015-08-01, 2015-09-01, 2015-10-01, 2015-11-01}, xticklabel={\pgfcalendar{tickcal}{\tick}{\tick}{\pgfcalendarshorthand{m}{.}}}, legend columns = 3]
    \addplot[total, red] table[x=time, y=value.Fixed.Weights] {plot-data/comparison_bonus.txt};
    \addlegendentry{Fixed Weights}
    \addplot[total, blue] table[x=time, y=value.Black.Litterman] {plot-data/comparison_bonus.txt};
    \addlegendentry{Black Litterman}
    \addplot[total, orange] table[x=time, y=value.Min.Variance.Robust] {plot-data/comparison_bonus.txt};
    \addlegendentry{Minimum Variance Robust}
    \addplot[total, black] table[x=time, y=value.MV.Max.Allocation...Short.Constraint] {plot-data/comparison_bonus.txt};
    \addlegendentry{Max Allocation No Short Sale}
    \addplot[total, green] table[x=time, y=value.Equal.Risk] {plot-data/comparison_bonus.txt};
    \addlegendentry{Equal Risk}
    \addplot[total, gray] table[x=time, y=value.MSCI] {plot-data/comparison_bonus.txt};
    \addlegendentry{MSCI}
  \end{axis}
\end{tikzpicture}


\chapter{Investment Philosophy \& Investment Approach}
\section*{Our Philosophy}
Supporters of passive investing believe in the implications of the CAPM and that markets are efficient, which means that market prices already fully reflect all fundamentals and therefore an outperformance cannot be achieved.
On the other hand, advocates of active investing believe that one can outperform the market.
Therefore it makes sense to consider timing and selection in the investment process.
JD investments believes that one can generate return by careful strategic asset allocation and consequent tactical asset allocation.
We believe that the advantages of active investing outweigh those of passive investing.
A passive investor can unwillingly be very concentrated even though he wants to exploit all diversification benefits.
That is the case if the market cap in an index increases during the time he holds the investment.
This means with active investing we see a diversification benefit.
 Especially as a consequence of the crisis people should be aware about the benefits of active investing.

An actively managed portfolio may have the chance of outperforming the market since it weights the assets differently and may not necessarily own all of the assets in the benchmark index.
It can also use special techniques such as shorting certain assets, which is not possible in a passive indexing strategy.
We as active portfolio managers want to understand our investors investment objectives and risk tolerance.
We use this knowledge to define an investment policy with risk and return goals, select a performance benchmark, choose an asset mix and allow for adjustments during the course of the market or changes in investment attitude of the investor.

By choosing the indices you wish to invest in and defining your risk tolerance as well as constraints you have already made your first step as an active investor.
JD Investments supports you in the next steps of active investing by providing sophisticated systematic solutions tailor made for your views and risk preferences.
 
\section*{Investment Approach}


\chapter*{M Squared}
Because our portfolios have different volatilities compared to the benchmark we use Modigliani Squared formula to obtain risk-adjusted returns of our portfolios. Afterwards, we could directly compare returns of our portfolios to those of the index and confirm that our portfolios outperform the benchmark not because of higher risks but because of high-level analysis and research that we conduct in order to find best – way solutions. To obtain $M^2$ returns, we scale Sharpe ratios of our portfolios by volatility of the benchmark and add risk-free rate. Since our risk-free rate is Euribor 6M, for this period of time it averaged to zero, so we need just scaling to obtain risk-adjusted return of our portfolios. In formula terms, it looks as follows:

\begin{align*}
  M^2 = S \cdot \sigma_{MSCI} = \mu_P \cdot \frac{\sigma_{MSCI}}{\sigma_P}
\end{align*}

$M^2$ has many benefits in comparison to Sharpe ratio as performance criteria.  First reason is that it could be difficult to compare portfolios with different Sharpe ratios, since the difference between them is not easily interpretable (we know that one Sharpe ratio is larger than another, but is this difference significant?). Secondly, it is hard to explain negative Sharpe ratios. Since we are always transparent to our clients we use M squared risk-adjusted returns, so that clients could compare returns of our portfolios directly to a benchmark and decide what suits them the best.


\chapter*{Conclusion \& Recommendation}
In order to make a judgement of how portfolios perform we need to take a look at the out-of-sample performance of the five portfolios we introduced. 
Compared to the Black Littermann portfolio the fixed weight portfolio has the higher rate of annual return of around 17.96\%. However, it also has higher volatility and a lower sharpe ratio.
The maximum drawdown is also higher lying at 9,08\%.
Both have a turnover of 0\% and therefore no transaction costs. 
The equal risk weighted portfolio has a lower annual return of 13.68\% compared to the Black Littermann and the fixed weight portfolio. It also has a lower volatility and maximum drawdown. With 1.49 it has a higher sharpe ratio than the latter two. Turnover is 11.53\% and therefore we have transaction costs of
0.06 \%. 
Of the five portfolios introduced the minimum variance (robust) portfolio has the highest attainable performance.
The minimum variance (robust) portfolio has a higher sharpe ratio than the fixed weight portfolio but lower than the equal risk and Black Littermann portfolio.
The maximum drawdown is lower than the Black LItterman and fixed weights portfolio but higher than the equal risk portfolio.
The turnover is higher than the Black Littermann and fixed weights portfolio but lower than the minimum variance(robust) portfolio.
Transaction costs amount to 0.05\%.
The MV max allocation and shortsale constraint portfolio has the lowest return compared to the other four portfolios and the lowest sharpe ratio. Therefore, it also has the lowest volatility and maximum drawdown.
Turnover is 12.10\% and transaction costs are 0.06\%. 

After comparing the performance metrics of the five portfolios we have constructed we believe that the equal risk portfolio is the best portfolio choice for you.
Even though it has a lower performance than the fixed weights, Black LIttermann and minimum variance(robust) portfolio we believe it fits best to your investment needs.
Of the five portfolios it has the highest sharpe ratio which means that it gives you the highest risk-return trade-off.
Secondly, it is the only portfolio that generates a positive alpha.
This means that it is evident that you are benefiting from us as your portfolio managers making investment decisions for you.
Our outperformance is also clear when comparing the performance to the respective benchmark. …XX


\chapter*{Code}
\lstinputlisting[language=R, basicstyle=\scriptsize]{../0-helper.R}

\lstinputlisting[language=R, basicstyle=\scriptsize]{../1-save_data.R}

\lstinputlisting[language=R, basicstyle=\scriptsize]{../1-data.R}

\lstinputlisting[language=R, basicstyle=\scriptsize]{../2-analysis.R}

\lstinputlisting[language=R, basicstyle=\scriptsize]{../2-application.R}

\lstinputlisting[language=R, basicstyle=\scriptsize]{../3-analysis2.R}

\lstinputlisting[language=R, basicstyle=\scriptsize]{../mdd.R}

\end{document}
